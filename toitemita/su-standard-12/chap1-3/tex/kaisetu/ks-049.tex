⑴<br>
 $1$□□□という形の整数の個数は,$5$ 個の数字から $3$ 個を取り出す順列の個数に等しいから,$_{5}\mathrm{P}_{3} = 60$ 個。<br>
 $20$□□,$21$□□という形の整数の個数は,$4$ 個の数字から $2$ 個を取り出す順列の個数に等しいから,それぞれ $_{4}\mathrm{P}_{3} = 12$ 個ずつある。<br>
 $230$□,$231$□という形の整数の個数は,それぞれ $3$ 個ずつある。<br>
 これらの整数の後,$2340$,$2341$,$2345$ という順に整数が並ぶ。<br>
 よって,$2345$ は<br>
  $60 + 12 \times 2 + 3 \times 2 + 3 = 93 \;\text{番目の整数である。}$ $\\[1.5em]$<br>

⑵<br>
 数字の並べ方を整理すると<br>
  異なる $5$ 個の数字から $3$ 個の数字を取り出す順列は,$_{5}\mathrm{P}_{3} = 60$ 通り<br>
  異なる $4$ 個の数字から $2$ 個の数字を取り出す順列は,$_{4}\mathrm{P}_{2} = 12$ 通り<br>
  異なる $3$ 個の数字から $1$ 個の数字を取り出す順列は,$3$ 通り $\\[1.5em]$<br>
 並べられる整数をはじめから順番に数え上げていくと<br>
  $1$□□□という形の整数が $60$ 個<br>
  $2$□□□という形の整数が $60$ 個<br>
  $3$□□□という形の整数が $60$ 個<br>
  $40$□□という形の整数が $12$ 個<br>
  $410$□という形の整数が $3$ 個<br>
  $412$□という形の整数が $3$ 個<br>
 であり,ここまでに現れた整数は全部で<br>
  $60 \times 3 + 12 + 3 \times 2 = 198 \;\text{個}$ <br>
 よって,$199$ 番目が $4130$ であり,$200$ 番目は $4132$ である。$\\[1.5em]$<br>

⑶<br>
 異なる $5$ 個の数字から $3$ 個の数字を取り出す順列は,$_{5}\mathrm{P}_{3} = 60$ 通りであるから<br>
  $1$□□□という形の整数が $60$ 個<br>
  $2$□□□という形の整数が $60$ 個<br>
並んだ後,$3$□□□という形の整数が現れる。<br>
 よって,初めて $3000$ を超えるのは $60 \times 2 + 1 = 121$ 番目であり,その整数は $3012$ である。 $\\[1.5em]$<br>