⑴<br>
 $7$ 個の数字から $4$ 個を取り出す順列であるから<br>
  $_{7}\mathrm{P}_{4} = 840 \;\text{個}$ $\\[1.5em]$<br>

⑵<br>
 一の位の数字の選び方は,$2$,$4$,$6$ の $3$ 通り。<br>
 そのそれぞれに対して,他の $3$ つの位の数字の選び方は,残り $6$ 個の数字から $3$ 個を取り出す順列であるから,$_{6}\mathrm{P}_{3}$ 通り。<br>
 よって,求める整数の個数は<br>
  $3 \times _{6}\mathrm{P}_{3} = 360 \;\text{個}$ $\\[1.5em]$<br>

⑶<br>
 一の位の数字の選び方は,$1$,$3$,$5$,$7$ の $4$ 通り。<br>
 そのそれぞれに対して,他の $3$ つの位の数字の選び方は,残り $6$ 個の数字から $3$ 個を取り出す順列であるから,$_{6}\mathrm{P}_{3}$ 通り。<br>
 よって,求める整数の個数は<br>
  $4 \times _{6}\mathrm{P}_{3} = 480 \;\text{個}$ $\\[1.5em]$<br>

⑷<br>
 一の位の数字の選び方は $5$ の $1$ 通り。<br>
 他の $3$ つの位の数字の選び方は,残り $6$ 個の数字から $3$ 個を取り出す順列であるから,$_{6}\mathrm{P}_{3}$ 通り。<br>
 よって,求める整数の個数は<br>
  $1 \times _{6}\mathrm{P}_{3} = 120 \;\text{個}$ $\\[1.5em]$<br>
