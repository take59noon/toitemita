⑴<br>
 $\mathrm{A}$ から $\mathrm{E}$ までの $5$ 箇所に対して,順番に色を割り当てていく。<br>
 よって,求める場合の数は<br>
  $5! = 120 \;\text{通り}$ $\\[1.5em]$<br>

⑵<br>
 $\mathrm{A}$ からアルファベット順に色を塗り進めるとする。<br>
 $\mathrm{A}$,$\mathrm{B}$,$\mathrm{C}$ はすべて互いに異なる色で塗り分ける必要がある。<br>
 $\mathrm{D}$ に塗る色は,$\mathrm{A}$ と $\mathrm{C}$ で塗った色以外であればよい。<br>
 そこで,$\mathrm{D}$ に塗る色が $\mathrm{B}$ で塗った色である場合と,$\mathrm{B}$ で塗った色以外である場合に分けて考える。$\\[1.5em]$<br>
 ⒜ $\mathrm{D}$ に塗る色が $\mathrm{B}$ で塗った色である場合<br>
  このとき,$\mathrm{E}$ に塗る色は,$\mathrm{A}$,$\mathrm{B}$ で塗った色以外であればよい。<br>
  したがって,<br>
   $\mathrm{A}$,$\mathrm{B}$,$\mathrm{C}$ の塗り方は,全部で $_{5}\mathrm{P}_{3}$ 通り<br>
   そのそれぞれに対して,$\mathrm{D}$ の塗り方は,$1$ 通り<br>
   そのそれぞれに対して,$\mathrm{E}$ の塗り方は,$3$ 通り<br>
 であるから,この場合における塗り方は<br>
   $_{5}\mathrm{P}_{3} \times 1 \times 3 = 180 \; \text{通り}$ $\\[1.5em]$<br>
 ⒝ $\mathrm{D}$ に塗る色が $\mathrm{B}$ で塗った色以外である場合<br>
  このとき,$\mathrm{E}$ に塗る色は,$\mathrm{A}$,$\mathrm{B}$,$\mathrm{D}$ で塗った色以外であればよい。<br>
  したがって,<br>
   $\mathrm{A}$,$\mathrm{B}$,$\mathrm{C}$ の塗り方は,全部で $_{5}\mathrm{P}_{3}$ 通り<br>
   そのそれぞれに対して,$\mathrm{D}$ の塗り方は,$2$ 通り<br>
   そのそれぞれに対して,$\mathrm{E}$ の塗り方は,$2$ 通り<br>
 であるから,この場合における塗り方は<br>
   $_{5}\mathrm{P}_{3} \times 2 \times 2 = 240 \; \text{通り}$ $\\[1.5em]$<br>
 以上より,求める場合の数は<br>
  $180 + 240 = 420 \;\text{通り}$ $\\[1.5em]$<br>