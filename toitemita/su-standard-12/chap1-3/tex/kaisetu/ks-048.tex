⑴<br>
 $\mathrm{SHIKEN}$ の $6$ つの文字をアルファベット順に並べると,$\mathrm{E}$,$\mathrm{H}$,$\mathrm{I}$,$\mathrm{K}$,$\mathrm{N}$,$\mathrm{S}$ であり,この中に重複する文字はない。<br>
 ここで,異なる $4$ 文字を $1$ 列に並べる並べ方は,$4! = 24$ 通りであるから,$\mathrm{EH}\text{□□□□}$という形の文字列は,全部で $24$ 個ある。<br>
 よって,$25$ 番目の文字列は,$\mathrm{EIHKNS}$ $\\[1.5em]$<br>

⑵<br>
 $\mathrm{SHIKEN}$ の $6$ つの文字をアルファベット順に並べると,$\mathrm{E}$,$\mathrm{H}$,$\mathrm{I}$,$\mathrm{K}$,$\mathrm{N}$,$\mathrm{S}$ であり,この中に重複する文字はない。<br>
 また,異なる文字を $1$ 列に並べる並べ方は,以下の通りである。<br>
  異なる $3$ 文字を $1$ 列に並べる並べ方は,$3! = 6$ 通り<br>
  異なる $4$ 文字を $1$ 列に並べる並べ方は,$4! = 24$ 通り<br>
  異なる $5$ 文字を $1$ 列に並べる並べ方は,$5! = 120$ 通り $\\[1.5em]$<br>
 $\mathrm{SHIKEN}$ よりも前に並ぶ文字列を順に数え上げていくと<br>
  $\mathrm{E}\text{□□□□□}$という形の文字列が $120$ 個<br>
  $\mathrm{H}\text{□□□□□}$という形の文字列が $120$ 個<br>
  $\mathrm{I}\text{□□□□□}$という形の文字列が $120$ 個<br>
  $\mathrm{K}\text{□□□□□}$という形の文字列が $120$ 個<br>
  $\mathrm{N}\text{□□□□□}$という形の文字列が $120$ 個<br>
  $\mathrm{SE}\text{□□□□}$という形の文字列が $24$ 個<br>
  $\mathrm{SHE}\text{□□□}$という形の文字列が $6$ 個<br>
であり,この後,$\mathrm{SHIEKN}$,$\mathrm{SHIENK}$,$\mathrm{SHIKEN}$ という順に文字列が並ぶ。<br>
 よって,$\mathrm{SHIKEN}$ は<br>
  $120 \times 5 + 24 + 6 + 3 = 633 \;\text{番目}$ <br>
の文字列である。 $\\[1.5em]$<br>