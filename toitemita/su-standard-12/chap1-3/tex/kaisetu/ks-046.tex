⑴<br>
 千の位の数字の選び方は,$0$ を除いた $6$ 通り。<br>
 そのそれぞれに対して,他の位の数字の選び方は,残り $6$ 個の数字から $3$ 個を取り出す順列であるから,$_{6}\mathrm{P}_{3}$ 通り。<br>
 よって,求める整数の個数は<br>
  $6 \times _{6}\mathrm{P}_{3} = 720 \;\text{個}$ $\\[1.5em]$<br>

⑵<br>
 一の位の数字が $0$ である場合と,$0$ 以外の偶数である場合に分けて考える。$\\[1.5em]$<br>
 ⒜ 一の位の数字が $0$ である場合<br>
  一の位の数字の選び方は $0$ の $1$ 通り。<br>
  他の $3$ つの位の数字の選び方は,残り $6$ 個の数字から $3$ 個を取り出す順列であるから,$_{6}\mathrm{P}_{3}$ 通り。<br>
  したがって<br>
   $1 \times _{6}\mathrm{P}_{3} = 120 \;\text{個}$ $\\[1.5em]$<br>
 ⒝ 一の位の数字が $0$ 以外の偶数である場合<br>
  一の位の数字の選び方は, $2$,$4$,$6$ の $3$ 通り。<br>
  そのそれぞれに対して,千の位の数字の選び方は一の位の数字と $0$ を除いて $5$ 通りあり,百の位と十の位の数字の選び方は残り $5$ 個の数字から $2$ 個を取り出す順列であるから,$_{5}\mathrm{P}_{2}$ 通り。<br>
  したがって<br>
   $3 \times 5 \times _{5}\mathrm{P}_{2} = 300 \;\text{個}$ $\\[1.5em]$<br>
 以上より,求める整数の個数は<br>
  $120 + 300 = 420 \;\text{個}$ $\\[1.5em]$<br>

⑶<br>
 求める個数は,すべての整数の個数から,偶数の個数を取り除いたものである。<br>
 すべての整数の個数は,⑴より,$720$ 通り。<br>
 偶数の個数は,⑵より,$420$ 通り。<br>
 よって,求める整数の個数は<br>
  $720 - 420 =300 \;\text{個}$ $\\[1.5em]$<br>

⑷<br>
 千の位の数字が $3$ である場合と,$4$ 以上の数字である場合に分けて考える。$\\[1.5em]$<br>
 ⒜ 千の位の数字が $3$ である場合<br>
  千の位の数字は $3$ の $1$ 通り。<br>
  百の位の数字は,$2$,$4$,$5$,$6$ の $4$ 通り。<br>
  残りの2桁の数字は,$5$ 個の数字から $2$ 個を取り出す順列であるから,$_{5}\mathrm{P}_{2}$ 通り。<br>
  したがって<br>
   $1 \times 4 \times _{5}\mathrm{P}_{2} = 80 \;\text{個}$ $\\[1.5em]$<br>
 ⒝ 千の位の数字が $4$ 以上の数字である場合<br>
  千の位の数字の選び方は,$4$,$5$,$6$ の $3$ 通り。<br>
  そのそれぞれに対して,他の $3$ つの位の数字の選び方は,残り $6$ 個の数字から $3$ 個を取り出す順列であるから,$_{6}\mathrm{P}_{3}$ 通り。<br>
  したがって<br>
   $3 \times _{6}\mathrm{P}_{3} = 360 \;\text{個}$ $\\[1.5em]$<br>
 以上より,求める整数の個数は<br>
  $80 + 360 = 520 \;\text{個}$ $\\[1.5em]$<br>
