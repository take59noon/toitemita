 全体集合を $U$ とおくと,$n(U)=100$ $\\[1em]$<br>

⑴<br>
 $4$ で割り切れる数全体の集合を $A$ とおくと
$$
A = \{4 \times 1,\; 4 \times 2,\; \cdots ,\; 4 \times 25\}
$$
 よって,$n(A) = 25$ $\\[1em]$<br>

⑵<br>
 $6$ で割り切れる数全体の集合を $B$ とおくと
$$
B = \{6 \times 1,\; 6 \times 2,\; \cdots ,\; 6 \times 16\}
$$
 したがって,$n(B) = 16$ より
\begin{eqnarray*}
  n\rpln{\myoverline{B}} &=& n(U) - n(B) \\[0.6em]
                       &=& 100 - 16 = 84 
\end{eqnarray*}

⑶<br>
 $4$ でも $6$ でも割り切れる数,すなわち $12$ で割り切れる数全体の集合は
$$
A \cap B = \{12 \times 1,\; 12 \times 2,\; \cdots ,\; 12 \times 8\}
$$
であるから, $n(A \cap B) = 8$<br>
 $4$ と $6$ の少なくとも一方で割り切れる数全体の集合は,$A \cup B$<br>
 よって,求める数の個数は
\begin{eqnarray*}
  n(A \cup B) &=& n(A) + n(B) - n(A \cap B) \\[0.6em]
                   &=& 25 + 16 - 8 = 33
\end{eqnarray*}
  
⑷<br>
 $4$ でも $6$ でも割り切れない数全体の集合は,$\myoverline{A} \cap \myoverline{B}$<br>
 ド・モルガンの法則より
$$
\myoverline{A} \cap \myoverline{B} = \myoverline{A \cup B}
$$
 よって,求める数の個数は
\begin{eqnarray*}
  n\rpln{\myoverline{A \cup B}} &=& n(U) - n(A \cup B) \\[0.6em]
                              &=& 100 - 33 = 67
\end{eqnarray*}