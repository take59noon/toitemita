 全体集合を $U$ とおくと,$n(U)=1000$ $\\[1em]$<br>

⑴<br>
 $3$ の倍数全体の集合を $A$ とおくと
$$
A = \{3 \times 1,\; 3 \times 2,\; \cdots ,\; 3 \times 333\}
$$
 よって,$n(A) = 333$ $\\[1em]$<br>

⑵<br>
 $5$ の倍数全体の集合を $B$ とおくと
$$
B = \{5 \times 1,\; 5 \times 2,\; \cdots ,\; 5 \times 200\}
$$
 よって,$n(B) = 200$ $\\[1em]$<br>

⑶<br>
 $3$ の倍数かつ $5$ の倍数,すなわち $15$ の倍数全体の集合は
$$
A \cap B = \{15 \times 1,\; 15 \times 2,\; \cdots ,\; 15 \times 66\}
$$
 よって, $n(A \cap B) = 66$ $\\[1em]$<br>
  
⑷<br>
 $3$ の倍数または $5$ の倍数全体の集合は,$A \cup B$ <br>
 よって,求める数の個数は
\begin{eqnarray*}
  n(A \cup B) &=& n(A) + n(B) - n(A \cap B) \\[0.6em]
                   &=& 333 + 200 - 66 = 467
\end{eqnarray*}

⑸<br>
 求める数の個数は
\begin{eqnarray*}
  n(\myoverline{A}) &=& n(U) - n(A) \\[0.6em]
                  &=& 1000 - 333 = 667
\end{eqnarray*}

⑹<br>
 $3$ の倍数でなく$5$ の倍数でもない数全体の集合は,$\myoverline{A} \cap \myoverline{B}$<br>
 ド・モルガンの法則より
$$
\myoverline{A} \cap \myoverline{B} = \myoverline{A \cup B}
$$
 よって,求める数の個数は
\begin{eqnarray*}
  n\rpln{\myoverline{A \cup B}} &=& n(U) - n(A \cup B) \\[0.6em]
                              &=& 1000 - 467 = 533
\end{eqnarray*}

⑺<br>
 $3$ の倍数であるが $5$ の倍数でない数全体の集合は,$A \cap \myoverline{B}$ <br>
 よって,求める数の個数は
\begin{eqnarray*}
  n\rpln{A \cap \myoverline{B}} &=& n(A) - n(A \cap B) \\[0.6em]
                               &=& 333 - 66 = 267
\end{eqnarray*}
