 かぜ薬を携帯した人全体の集合を $A$,
胃薬を携帯した人全体の集合を $B$ とおく。<br>
 また,全体集合を $U$ とおく。$\\[1em]$<br>

⑴<br>
 かぜ薬と胃薬を両方とも携帯した人全体の集合は,$n(A \cap B)$ $\\[1em]$<br>
 $n(A \cap B)$ が最大値をとるのは,
$n(A) < n(B)$ より $A \subset B$ のときである。<br>
 このとき
$$
n(A \cap B) = n(A) = 75
$$

 $n(A \cap B)$ が最小値をとるのは,
$n(A)+n(B)>n(U)$ より $A \cup B = U$ のときである。<br>
 このとき,$n(A \cup B) = n(A) + n(B) - n(A \cap B)$ より
\begin{eqnarray*}
  n(A \cap B) &=& n(A) + n(B) - n(A \cup B) \\[0.6em]
              &=& n(A) + n(B) - n(U) \\[0.6em]
              &=& 75 + 80 - 100 = 55
\end{eqnarray*}

⑵<br>
 かぜ薬と胃薬を両方とも携帯していない人全体の集合は,
$n\rpln{\myoverline{A} \cap \myoverline{B}}$ <br>
 ド・モルガンの法則より
$$
\myoverline{A} \cap \myoverline{B} = \myoverline{A \cup B}
$$
であるから
\begin{eqnarray*}
  n\rpln{\myoverline{A} \cap \myoverline{B}} 
    &=& n\rpln{\myoverline{A \cup B}} \\[0.6em]
    &=& n(U) - n(A \cup B)
\end{eqnarray*}

したがって,<br>
 $n\rpln{\myoverline{A} \cap \myoverline{B}}$ が最大となるのは,
$n(A \cup B)$ が最小,すなわち$A \cup B = B$ のときであるから
\begin{eqnarray*}
  n\rpln{\myoverline{A} \cap \myoverline{B}} 
    &=& n(U) - n(A \cup B) \\[0.6em]
    &=& n(U) - n(B) \\[0.6em]
    &=& 100 - 80 = 20
\end{eqnarray*}

 $n\rpln{\myoverline{A} \cap \myoverline{B}}$ が最小となるのは,
$n(A \cup B)$ が最大,すなわち$A \cup B = U$ のときであるから
\begin{eqnarray*}
  n\rpln{\myoverline{A} \cap \myoverline{B}} 
    &=& n(U) - n(A \cup B) \\[0.6em]
    &=& n(U) - n(U) \\[0.6em]
    &=& 0
\end{eqnarray*}