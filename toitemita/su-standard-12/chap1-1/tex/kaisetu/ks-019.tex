 $100$ から $200$ までの整数のうち,$3$ で割り切れる数全体の集合を $A$,
$7$ で割り切れる数全体の集合を $B$ とおくと
\begin{eqnarray*}
  && A = \{3 \times 34,\; 3 \times 35,\; \cdots ,\; 3 \times 66\} \\[0.6em]
  && B = \{3 \times 15,\; 3 \times 16,\; \cdots ,\; 3 \times 28\} \\[0.6em]
\end{eqnarray*}
であるから
\begin{eqnarray*}
  && n(A) = 66 - 34 + 1 = 33 \\[0.6em]
  && n(B) = 28 - 15 + 1 = 14 \\[0.6em]
\end{eqnarray*}
 また,全体集合を $U$ とおくと
$$
n(U) = 200 - 100 + 1 = 101
$$

⑴<br>
 $3$ でも $7$ でも割り切れない数全体の集合は,$\myoverline{A} \cap \myoverline{B}$<br>
 ド・モルガンの法則より
\begin{eqnarray*}
  \myoverline{A} \cap \myoverline{B} 
  &=& \myoverline{A \cup B} \\[0.6em]
  &=& n(U) - n(A \cup B) \\[0.6em]
  &=& n(U) - \rpln[\big]{n(A) + n(B) - n(A \cap B)} \cdots\cdots\text{①}
\end{eqnarray*}
 ここで,集合 $A \cap B$ について考える。<br>
 集合 $A \cap B$ は,$3$ でも $7$ でも割り切れる数,
すなわち $21$ で割り切れる数全体の集合であるから
$$
A \cap B = \{21 \times 5,\; 21 \times 6,\; \cdots ,\; 21 \times 9\}
$$
 したがって
$$
n(A \cap B) = 9 - 5 + 1 = 5
$$
 よって,①に $n(U)$,$n(A)$,$n(B)$,$n(A \cap B)$ をそれぞれ代入して
$$
n\rpln{\myoverline{A} \cap \myoverline{B}} = 101 - (33+14-5) = 59
$$

⑵<br>
 $3$ で割り切れるが $7$ で割り切れない数全体の集合は,$A \cap \myoverline{B}$ <br>
 よって
\begin{eqnarray*}
  n\rpln{A \cap \myoverline{B}} &=& n(A) - n(A \cap B) \\[0.6em]
                                &=& 33 - 5 = 28 
\end{eqnarray*}
