 数学の合格者全体の集合を $A$,
英語の合格者全体の集合を $B$ とおくと
\begin{eqnarray*}
  && n(A) = 30 \\[0.6em]
  && n(B) = 50 \\[0.6em]
  && n\rpln{\myoverline{A} \cap \myoverline{B}} = 8
\end{eqnarray*}
 また,全体集合を $U$ とおくと,$n(U)=60$ $\\[1em]$<br>

⑴<br>
 $2$ 科目とも合格した人全体の集合は,$A \cap B$ <br>
 ここで
$$
n(A \cup B) = n(A) + n(B) - n(A \cap B)
$$
を変形すると
$$
n(A \cap B) = n(A) + n(B) - n(A \cup B)
$$
であるから,$n(A \cup B)$ がわかれば $n(A \cap B)$ が得られる。<br>
<br>
 ド・モルガンの法則より
$$
\myoverline{A} \cap \myoverline{B} = \myoverline{A \cup B}
$$
であるから
\begin{eqnarray*}
  n(A \cup B) &=& n(U) - n\rpln{\myoverline{A \cup B}} \\[0.6em]
              &=& n(U) - n\rpln{\myoverline{A} \cap \myoverline{B}} \\[0.6em]
              &=& 60 - 8 = 52
\end{eqnarray*}
 よって,求める人数は
\begin{eqnarray*}
  n(A \cap B) &=& n(A) + n(B) - n(A \cup B) \\[0.6em]
              &=& 30 + 50 - 52 = 28
\end{eqnarray*}

⑵<br>
 数学だけ合格した人全体の集合は,$A \cap \myoverline{B}$ <br>
 よって,求める人数は
\begin{eqnarray*}
  n\rpln{A \cap \myoverline{B}} &=& n(A) - n(A \cap B) \\[0.6em]
                              &=& 30 - 28 = 2
\end{eqnarray*}