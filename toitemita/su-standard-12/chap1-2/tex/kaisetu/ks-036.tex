 $\mbox{$10$ 円,$30$ 円,$70$ 円のあめ玉を買う個数を,順に$x,\myquad y,\myquad z$ とすると}$ <br>
  $10x+30y+70z=230$ <br>
  すなわち,$x+3y+7z=23 \quad \cdots \text{①}$ <br>
 ①を満たす自然数 $x,\myquad y,\myquad z$ の組の個数を求めればよい。$\\[1em]$<br>

 ①を変形すると<br>
\begin{eqnarray*}
  && 7z = 23 - (x + 3y) \\[.5em]
  &&  z = \dfrac{23 - (x + 3y)}{7} 
\end{eqnarray*}
 $x,\myquad y$ は自然数であるから
\begin{eqnarray*}
  z = \dfrac{23 - (x + 3y)}{7} \leqq \dfrac{23 - (1 + 3 \cdot 1)}{7} = \dfrac{19}{7}
\end{eqnarray*}
 $z$ は自然数であるから,$z = 1,\myquad 2$ $\\[1em]$<br>

 ⒜$\myquad z=1$ のとき<br>
  与式は $x+3y=16$ となるから
\begin{eqnarray*}
  \quad y = \dfrac{16-x}{3}
\end{eqnarray*}
  $x$ は自然数であるから
\begin{eqnarray*}
  \quad y \leqq \dfrac{16-1}{3} = 5
\end{eqnarray*}
  $y$ は自然数であるから,$y = 1,\myquad 2,\myquad 3,\myquad 4,\myquad 5$ <br>
  よって<br>
   $(x,\myquad y) = (13,\myquad 1),\myquad (10,\myquad 2),\myquad (7,\myquad 3),\myquad (4,\myquad 4),\myquad (1,\myquad 5)$ <br>
 となり,$5$ 通り。$\\[1em]$<br>

 ⒝$\myquad z=2$ のとき<br>
  与式は $x+3y=9$ となるから
\begin{eqnarray*}
  \quad y = \dfrac{9-x}{3}
\end{eqnarray*}
  $x$ は自然数であるから
\begin{eqnarray*}
  \quad y \leqq \dfrac{9-1}{3} = \dfrac{8}{3} 
\end{eqnarray*}
  $y$ は自然数であるから,$y = 1,\myquad 2$ <br>
  よって<br>
   $(x,\myquad y) = (6,\myquad 1),\myquad (3,\myquad 2)$ <br>
 となり,$2$ 通り。$\\[1em]$<br>

 以上より,求める場合の数は <br>
  $5+2=7$ 通り