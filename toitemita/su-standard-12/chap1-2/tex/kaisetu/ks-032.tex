⑴<br>
 支払うことができる金額は <br>
  $500$ 円硬貨 $0$ 枚から $2$ 枚 <br>
  $100$ 円硬貨 $0$ 枚から $3$ 枚 <br>
  $10$ 円硬貨 $0$ 枚から $4$ 枚 <br>
を使用してできる金額の場合の数から,どの硬貨を $1$ 枚も使用しない場合の数を取り除いたものである。<br>
 よって,求める場合の数は<br>
  $3 \times 4 \times 5 - 1 = 59$ 通り $\\[1em]$<br>

⑵<br>
 $100$ 円硬貨 $5$ 枚で $500$ 円硬貨 $1$ 枚分となる。<br>
 以下の $2$ つの場合に分けて考える。<br>
  ⒜$\myquad 500$ 円硬貨の使用が $4$ 枚分であるとき<br>
  ⒝$\myquad 500$ 円硬貨の使用が $3$ 枚分以下であるとき $\\[1em]$<br>
 ⒜$\myquad 500$ 円硬貨の使用が $4$ 枚分であるとき<br>
  $100$ 円硬貨 $5$ 枚を $500$ 円硬貨 $1$ 枚分として使用する。<br>
  残りの $100$ 円硬貨の枚数は $2$ 枚である。<br>
  このとき,支払うことができる金額は<br>
   $500$ 円硬貨 $4$ 枚 <br>
   $100$ 円硬貨 $0$ 枚から $2$ 枚 <br>
   $10$ 円硬貨 $0$ 枚から $3$ 枚 <br>
 を使用してできる金額であるから<br>
   $1 \times 3 \times 4 = 12$ 通り $\\[1em]$<br>
 ⒝$\myquad 500$ 円硬貨の使用が $3$ 枚分以下であるとき<br>
  このとき,支払うことができる金額は<br>
   $500$ 円硬貨 $0$ 枚から $3$ 枚 <br>
   $100$ 円硬貨 $0$ 枚から $4$ 枚 <br>
   $10$ 円硬貨 $0$ 枚から $3$ 枚 <br>
 を使用してできる金額の場合の数から,どの硬貨を $1$ 枚も使用しない場合の数を取り除いたものであるから<br>
   $4 \times 5 \times 4 - 1 = 79$ 通り $\\[1em]$<br>
 以上より,求める場合の数は<br>
  $12 + 79 = 91$ 通り
