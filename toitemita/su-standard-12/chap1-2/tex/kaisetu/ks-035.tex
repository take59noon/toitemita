 最も係数の大きい $z$ から順番に,値を絞り込んでいけばよい。$\\[1em]$<br>

 与式を変形すると<br>
\begin{eqnarray*}
  && 3z = 12 - (x + 2y) \\[.5em]
  &&  z = 4 - \dfrac{x + 2y}{3}
\end{eqnarray*}
 $x,\myquad y$ は自然数であるから
\begin{eqnarray*}
  z = 4 - \dfrac{x + 2y}{3} \leqq 4 - \dfrac{1+2 \cdot 1}{3} = 3
\end{eqnarray*}
 $z$ は自然数であるから,$z = 1,\myquad 2,\myquad 3$ $\\[1em]$<br>

 ⒜$\myquad z=1$ のとき<br>
  与式は $x+2y=9$ となるから
\begin{eqnarray*}
  \quad y = \dfrac{9-x}{2}
\end{eqnarray*}
  $x$ は自然数であるから
\begin{eqnarray*}
  \quad y \leqq \dfrac{9-1}{2} =4
\end{eqnarray*}
  $y$ は自然数であるから,$y = 1,\myquad 2,\myquad 3,\myquad 4$ <br>
  よって<br>
   $(x,\myquad y) = (7,\myquad 1),\myquad (5,\myquad 2),\myquad (3,\myquad 3),\myquad (1,\myquad 4)$ <br>
 となり,$4$ 通り。$\\[1em]$<br>

 ⒝$\myquad z=2$ のとき<br>
  与式は $x+2y=6$ となるから
\begin{eqnarray*}
  \quad y = \dfrac{6-x}{2}
\end{eqnarray*}
  $x$ は自然数であるから
\begin{eqnarray*}
  \quad y \leqq \dfrac{6-1}{2} = \dfrac{5}{2}
\end{eqnarray*}
  $y$ は自然数であるから,$y = 1,\myquad 2$ <br>
  よって<br>
   $(x,\myquad y) = (4,\myquad 1),\myquad (2,\myquad 2)$ <br>
 となり,$2$ 通り。$\\[1em]$<br>

 ⒞$\myquad z=3$ のとき<br>
  与式は $x+2y=3$ となるから
\begin{eqnarray*}
  \quad y = \dfrac{3-x}{2}
\end{eqnarray*}
  $x$ は自然数であるから
\begin{eqnarray*}
  y \leqq \dfrac{3-1}{2} = 1
\end{eqnarray*}
  $y$ は自然数であるから,$y = 1$ <br>
  よって<br>
   $(x,\myquad y) = (1,\myquad 1)$ <br>
 となり,$1$ 通り。$\\[1em]$<br>

 以上より,等式 $x+2y+3z=12$ を満たす自然数の組は <br>
  $4+2+1 =7$ 組